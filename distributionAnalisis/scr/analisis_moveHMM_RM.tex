% Options for packages loaded elsewhere
\PassOptionsToPackage{unicode}{hyperref}
\PassOptionsToPackage{hyphens}{url}
%
\documentclass[
]{article}
\usepackage{amsmath,amssymb}
\usepackage{lmodern}
\usepackage{iftex}
\ifPDFTeX
  \usepackage[T1]{fontenc}
  \usepackage[utf8]{inputenc}
  \usepackage{textcomp} % provide euro and other symbols
\else % if luatex or xetex
  \usepackage{unicode-math}
  \defaultfontfeatures{Scale=MatchLowercase}
  \defaultfontfeatures[\rmfamily]{Ligatures=TeX,Scale=1}
\fi
% Use upquote if available, for straight quotes in verbatim environments
\IfFileExists{upquote.sty}{\usepackage{upquote}}{}
\IfFileExists{microtype.sty}{% use microtype if available
  \usepackage[]{microtype}
  \UseMicrotypeSet[protrusion]{basicmath} % disable protrusion for tt fonts
}{}
\makeatletter
\@ifundefined{KOMAClassName}{% if non-KOMA class
  \IfFileExists{parskip.sty}{%
    \usepackage{parskip}
  }{% else
    \setlength{\parindent}{0pt}
    \setlength{\parskip}{6pt plus 2pt minus 1pt}}
}{% if KOMA class
  \KOMAoptions{parskip=half}}
\makeatother
\usepackage{xcolor}
\usepackage[margin=1in]{geometry}
\usepackage{color}
\usepackage{fancyvrb}
\newcommand{\VerbBar}{|}
\newcommand{\VERB}{\Verb[commandchars=\\\{\}]}
\DefineVerbatimEnvironment{Highlighting}{Verbatim}{commandchars=\\\{\}}
% Add ',fontsize=\small' for more characters per line
\usepackage{framed}
\definecolor{shadecolor}{RGB}{248,248,248}
\newenvironment{Shaded}{\begin{snugshade}}{\end{snugshade}}
\newcommand{\AlertTok}[1]{\textcolor[rgb]{0.94,0.16,0.16}{#1}}
\newcommand{\AnnotationTok}[1]{\textcolor[rgb]{0.56,0.35,0.01}{\textbf{\textit{#1}}}}
\newcommand{\AttributeTok}[1]{\textcolor[rgb]{0.77,0.63,0.00}{#1}}
\newcommand{\BaseNTok}[1]{\textcolor[rgb]{0.00,0.00,0.81}{#1}}
\newcommand{\BuiltInTok}[1]{#1}
\newcommand{\CharTok}[1]{\textcolor[rgb]{0.31,0.60,0.02}{#1}}
\newcommand{\CommentTok}[1]{\textcolor[rgb]{0.56,0.35,0.01}{\textit{#1}}}
\newcommand{\CommentVarTok}[1]{\textcolor[rgb]{0.56,0.35,0.01}{\textbf{\textit{#1}}}}
\newcommand{\ConstantTok}[1]{\textcolor[rgb]{0.00,0.00,0.00}{#1}}
\newcommand{\ControlFlowTok}[1]{\textcolor[rgb]{0.13,0.29,0.53}{\textbf{#1}}}
\newcommand{\DataTypeTok}[1]{\textcolor[rgb]{0.13,0.29,0.53}{#1}}
\newcommand{\DecValTok}[1]{\textcolor[rgb]{0.00,0.00,0.81}{#1}}
\newcommand{\DocumentationTok}[1]{\textcolor[rgb]{0.56,0.35,0.01}{\textbf{\textit{#1}}}}
\newcommand{\ErrorTok}[1]{\textcolor[rgb]{0.64,0.00,0.00}{\textbf{#1}}}
\newcommand{\ExtensionTok}[1]{#1}
\newcommand{\FloatTok}[1]{\textcolor[rgb]{0.00,0.00,0.81}{#1}}
\newcommand{\FunctionTok}[1]{\textcolor[rgb]{0.00,0.00,0.00}{#1}}
\newcommand{\ImportTok}[1]{#1}
\newcommand{\InformationTok}[1]{\textcolor[rgb]{0.56,0.35,0.01}{\textbf{\textit{#1}}}}
\newcommand{\KeywordTok}[1]{\textcolor[rgb]{0.13,0.29,0.53}{\textbf{#1}}}
\newcommand{\NormalTok}[1]{#1}
\newcommand{\OperatorTok}[1]{\textcolor[rgb]{0.81,0.36,0.00}{\textbf{#1}}}
\newcommand{\OtherTok}[1]{\textcolor[rgb]{0.56,0.35,0.01}{#1}}
\newcommand{\PreprocessorTok}[1]{\textcolor[rgb]{0.56,0.35,0.01}{\textit{#1}}}
\newcommand{\RegionMarkerTok}[1]{#1}
\newcommand{\SpecialCharTok}[1]{\textcolor[rgb]{0.00,0.00,0.00}{#1}}
\newcommand{\SpecialStringTok}[1]{\textcolor[rgb]{0.31,0.60,0.02}{#1}}
\newcommand{\StringTok}[1]{\textcolor[rgb]{0.31,0.60,0.02}{#1}}
\newcommand{\VariableTok}[1]{\textcolor[rgb]{0.00,0.00,0.00}{#1}}
\newcommand{\VerbatimStringTok}[1]{\textcolor[rgb]{0.31,0.60,0.02}{#1}}
\newcommand{\WarningTok}[1]{\textcolor[rgb]{0.56,0.35,0.01}{\textbf{\textit{#1}}}}
\usepackage{graphicx}
\makeatletter
\def\maxwidth{\ifdim\Gin@nat@width>\linewidth\linewidth\else\Gin@nat@width\fi}
\def\maxheight{\ifdim\Gin@nat@height>\textheight\textheight\else\Gin@nat@height\fi}
\makeatother
% Scale images if necessary, so that they will not overflow the page
% margins by default, and it is still possible to overwrite the defaults
% using explicit options in \includegraphics[width, height, ...]{}
\setkeys{Gin}{width=\maxwidth,height=\maxheight,keepaspectratio}
% Set default figure placement to htbp
\makeatletter
\def\fps@figure{htbp}
\makeatother
\setlength{\emergencystretch}{3em} % prevent overfull lines
\providecommand{\tightlist}{%
  \setlength{\itemsep}{0pt}\setlength{\parskip}{0pt}}
\setcounter{secnumdepth}{-\maxdimen} % remove section numbering
\ifLuaTeX
  \usepackage{selnolig}  % disable illegal ligatures
\fi
\IfFileExists{bookmark.sty}{\usepackage{bookmark}}{\usepackage{hyperref}}
\IfFileExists{xurl.sty}{\usepackage{xurl}}{} % add URL line breaks if available
\urlstyle{same} % disable monospaced font for URLs
\hypersetup{
  pdftitle={Insights on the trajectories of harvesters using HMM},
  hidelinks,
  pdfcreator={LaTeX via pandoc}}

\title{Insights on the trajectories of harvesters using HMM}
\author{}
\date{\vspace{-2.5em}}

\begin{document}
\maketitle

In this document I will explain the main results and interpretations of
the analysis of the spatial trajectories of the harvesters using Hidden
Markov Models (HMM).

\hypertarget{i.-preparation-of-data}{%
\subsection{I. Preparation of data}\label{i.-preparation-of-data}}

We load ggplot2, dplyr, tidyverse and movehmm libraries used during the
analysis. We then load the data of the trajectories and do some punctual
modifications to it. We have in total 12 trajectories, with X, y
coordinates:

\begin{figure}
\centering
\includegraphics{analisis_moveHMM_RM_files/figure-latex/unnamed-chunk-3-1.pdf}
\caption{12 trajectories of harvesters. The green line represent the
Finca Hamburgo (H) and the blue line the Finca Irlanda (I)}
\end{figure}

In this first approach, we made no difference between the two farms for
the analysis of the states (but will use it a posteriori). A first
assumption for the analysis is made: we took these time irregular
trajectories (where each point represents one different tree, but where
the time between two trees is variable) and will treat them as a regular
trajectories. We only want to analyze if the movement of the workers
across the plots has different states (taking all the trajectories into
account). This also assumes that the underneath pattern of trees is
relatively homogeneous and is not the main determinant of the
trajectories. We convert this data base into a movehmm object:

\begin{verbatim}
## Movement data for 12 tracks:
## I_Ger1 -- 100 observations
## I_Ger2 -- 80 observations
## I_Mig3 -- 107 observations
## I_MigSam4 -- 76 observations
## H_Fran5 -- 77 observations
## H_Fran6 -- 70 observations
## H_Fran7 -- 115 observations
## H_Fran8 -- 63 observations
## H_Fran9 -- 97 observations
## I_Sam10 -- 104 observations
## H_Fran11 -- 88 observations
## I_Car12 -- 158 observations
## No covariates.
\end{verbatim}

\begin{Shaded}
\begin{Highlighting}[]
\FunctionTok{head}\NormalTok{(dataCosecha)}
\end{Highlighting}
\end{Shaded}

\begin{verbatim}
##       ID     step     angle   x   y
## 1 I_Ger1 3.605551        NA 146 117
## 2 I_Ger1 3.000000 -2.158799 143 115
## 3 I_Ger1 3.605551  2.553590 143 118
## 4 I_Ger1 2.236068 -1.446441 141 115
## 5 I_Ger1 5.000000  2.034444 139 116
## 6 I_Ger1 3.605551 -2.158799 139 111
\end{verbatim}

We can see that the distance step and the relative angle is calculated
for each of the 12 trajectories. We show the histograms of both the
steps and the angles. We separate colors for the different farms, but we
will treat it as the same (and then analyze the frequency of states per
farm).

\includegraphics{analisis_moveHMM_RM_files/figure-latex/unnamed-chunk-6-1.pdf}

\includegraphics{analisis_moveHMM_RM_files/figure-latex/unnamed-chunk-7-1.pdf}

Now, for the following analysis, we will only use the step distance and
treat both farms as one. We decide to not include the angles to simplify
the analysis and interpretation. In this sense, the input for the
analysis is the following histogram:

\includegraphics{analisis_moveHMM_RM_files/figure-latex/unnamed-chunk-8-1.pdf}

\hypertarget{ii.-movehmm-analysis}{%
\subsection{II. Movehmm analysis}\label{ii.-movehmm-analysis}}

The library movehmm uses HMM to separate trajectories into multiple
states (the hidden markov states) that are supposed to belong to
specific probability distributions (e.g.~gamma, lnorm, weibull) using
the ``emissions'' of the states (in this case, the step lengths or
angles). The algorithm estimates the most likely distributions of angles
and step length of each state, the transition probabilities and the
probabilty of each of the ``emissions'' to come from each of the states.

It uses:

\begin{itemize}
\tightlist
\item
  a predefined distribution with prior parameters (guesses)
\item
  a dataset of step lengths and angles (here we will only use the step
  lengths)
\item
  a predefined number of states
\end{itemize}

\hypertarget{a-priors-parameters}{%
\paragraph{a) Priors parameters}\label{a-priors-parameters}}

We used 3 different models for the step lengths to approach the
distributions of the states (weibull, lnorm and gamma). In order to
avoid local maximal likelihoods, we swapped across a range of prior
parameters for each model (1000). The min and max values of each range
were chosen using ** and trying to use distribution that could encompass
the histogram presented previously (gamma and weibull parameters). For
the weibull distribution, the maximum shape was considering the
left-skew and the scale was consider to include the 120 m steps. It is
important to note that these are only priors guesses, and the real
parameters can outbound these limits. Many these combinations of prior
parameters converge to the same final parameters.

\begin{verbatim}
##   value gamma.mean gamma.sd weibull.shape weibull.scale lnorm.location
## 1   min        0.1      0.1           0.0           0.1             -1
## 2   max       20.0     20.0           2.7          15.0             20
##   lnorm.scale
## 1       0.001
## 2       3.000
\end{verbatim}

\hypertarget{b-states-models-and-whole-loop}{%
\paragraph{b) States, models and whole
loop}\label{b-states-models-and-whole-loop}}

After this decision, we predefined 2 states and load the database. We
ran the models for the 1000 combinations of priors and the 3
distributions. For each model we register the minimum
negative-likelihood and the minimum AIC (both values should be
proportional, since all the models have the same number of parameters).

We plot the different AIC values for all the models and combinations of
parameters:

\includegraphics{analisis_moveHMM_RM_files/figure-latex/unnamed-chunk-12-1.pdf}

From this figure we note that in every model, almost all model generate
1 or 2 AIC values, but in the case of the gamma and lnorm distributions
we can see some specific outliers that have minimal AIC values (less
than 4500). Before taking these models for granted we decided to explore
its resulting parameters and biological meaning.

We filter the priors and resulting parameters for these outliers:

\begin{verbatim}
## # A tibble: 15 x 13
## # Groups:   model [2]
##    model prior_par0_st1 prior_par0_st2 prior_par1_st1 prior_par1_st2 minNegLike
##    <chr>          <dbl>          <dbl>          <dbl>          <dbl>      <dbl>
##  1 gamma           0.56          11.6            2.58           0.29       759.
##  2 gamma          10.4           10.3            8.86           0.44      1892.
##  3 gamma          10.6            2.7           11.2            8.92       191.
##  4 gamma           9.99           2.41          10.6            5.72      1989.
##  5 gamma           7.35           3.12           0.15          14.0       1806.
##  6 gamma           4.29           2.1            0.23          14.1        823.
##  7 gamma           9.03           1.17          13.8            0.13       759.
##  8 gamma           6.64           2.43           0.22          12.7       1916.
##  9 gamma           5.72          11.4            0.23          12.4        308.
## 10 lnorm           2.19          -0.48           2.91           0.79      2114.
## 11 lnorm           1.32           2.55           1.59           0.62      1621.
## 12 lnorm           1.31           1.2            1.26           0.05      1683.
## 13 lnorm           0.74           1.36           1.6            1.4       1775.
## 14 lnorm           0.99           0.05           1.47           1.22      2605.
## 15 lnorm           1.27           0.81           1.38           1.76      1537.
## # i 7 more variables: AIC_model <dbl>, st1_par0 <dbl>, st1_par1 <dbl>,
## #   st2_par0 <dbl>, st2_par1 <dbl>, conteo <dbl>, contador <dbl>
\end{verbatim}

If we plot 5 examples of the parameter 1 (blue dots as it represent a
measure of the mean) and the parameter 2 (blue line, as it represents a
measure of variance) for each of the model we get the following figure:

\includegraphics{analisis_moveHMM_RM_files/figure-latex/unnamed-chunk-15-1.pdf}

We note that in both cases, one of the states has near-zero variance and
the other state has non null variance that includes the other mean. This
second state encompasses all the steps of the trajectories except the
ones that have the exact value of the mean of the zero-variance state.
If we plot one example for the gamma and the lnorm distribution we get
the following figures:

For the gamma distribution (we show an example with 2 trajectories)

\begin{verbatim}
## Decoding states sequence... DONE
\end{verbatim}

\includegraphics[width=0.5\linewidth]{analisis_moveHMM_RM_files/figure-latex/unnamed-chunk-17-1}
\includegraphics[width=0.5\linewidth]{analisis_moveHMM_RM_files/figure-latex/unnamed-chunk-17-2}
\includegraphics[width=0.5\linewidth]{analisis_moveHMM_RM_files/figure-latex/unnamed-chunk-17-3}
\includegraphics[width=0.5\linewidth]{analisis_moveHMM_RM_files/figure-latex/unnamed-chunk-17-4}

For the lnorm distribution (we show an example with 2 trajectories)

\begin{verbatim}
## Decoding states sequence... DONE
\end{verbatim}

\includegraphics[width=0.5\linewidth]{analisis_moveHMM_RM_files/figure-latex/unnamed-chunk-18-1}
\includegraphics[width=0.5\linewidth]{analisis_moveHMM_RM_files/figure-latex/unnamed-chunk-18-2}
\includegraphics[width=0.5\linewidth]{analisis_moveHMM_RM_files/figure-latex/unnamed-chunk-18-3}
\includegraphics[width=0.5\linewidth]{analisis_moveHMM_RM_files/figure-latex/unnamed-chunk-18-4}

As we can see, the algorithm considers one state for the really small
steps (\textless4 m) and other state for all the other step lengths.
This is interesting but it doe not makes sense to assume on distribution
without any variance. In this sense, we assume that it makes sense to
remove these outliers.

If we plot the AIC figures without the outliers we have:
\includegraphics{analisis_moveHMM_RM_files/figure-latex/unnamed-chunk-19-1.pdf}

We observe some minimum values per model, that we are going to use.

\begin{Shaded}
\begin{Highlighting}[]
\NormalTok{DF\_TOTAL\_MIN }\OtherTok{\textless{}{-}}\NormalTok{ DF\_TOTAL\_1000 }\SpecialCharTok{\%\textgreater{}\%}
  \FunctionTok{filter}\NormalTok{(AIC\_model }\SpecialCharTok{\textgreater{}} \DecValTok{4500}\NormalTok{)}\SpecialCharTok{\%\textgreater{}\%}
  \FunctionTok{group\_by}\NormalTok{(model)}\SpecialCharTok{\%\textgreater{}\%}
  \FunctionTok{filter}\NormalTok{(AIC\_model }\SpecialCharTok{==} \FunctionTok{min}\NormalTok{(AIC\_model)) }\SpecialCharTok{\%\textgreater{}\%}
  \FunctionTok{group\_by}\NormalTok{(model, AIC\_model) }\SpecialCharTok{\%\textgreater{}\%}
  \FunctionTok{mutate}\NormalTok{(}\AttributeTok{conteo =} \DecValTok{1}\NormalTok{)}\SpecialCharTok{\%\textgreater{}\%}
  \FunctionTok{mutate}\NormalTok{(}\AttributeTok{contador=} \FunctionTok{cumsum}\NormalTok{(conteo))}\SpecialCharTok{\%\textgreater{}\%}
  \FunctionTok{filter}\NormalTok{(contador }\SpecialCharTok{==}\DecValTok{1}\NormalTok{)}
\end{Highlighting}
\end{Shaded}

\begin{Shaded}
\begin{Highlighting}[]
\NormalTok{modelos\_p }\OtherTok{\textless{}{-}} \FunctionTok{list}\NormalTok{(}\StringTok{"gamma"} \OtherTok{=} \DecValTok{0}\NormalTok{, }
     \StringTok{"weibull"} \OtherTok{=} \DecValTok{0}\NormalTok{,  }\DocumentationTok{\#\#segun el articulo}
     \StringTok{"lnorm"} \OtherTok{=}  \DecValTok{0}\NormalTok{)}

\ControlFlowTok{for}\NormalTok{ (i }\ControlFlowTok{in} \FunctionTok{seq}\NormalTok{(}\DecValTok{1}\NormalTok{,}\FunctionTok{dim}\NormalTok{(DF\_TOTAL\_MIN)[}\DecValTok{1}\NormalTok{],}\DecValTok{1}\NormalTok{))\{}
\NormalTok{  par0\_p }\OtherTok{\textless{}{-}} \FunctionTok{as.numeric}\NormalTok{(}\FunctionTok{c}\NormalTok{(DF\_TOTAL\_MIN}\SpecialCharTok{$}\NormalTok{prior\_par0\_st1[i], DF\_TOTAL\_MIN}\SpecialCharTok{$}\NormalTok{prior\_par0\_st2[i]))}
\NormalTok{  par1\_p }\OtherTok{\textless{}{-}} \FunctionTok{as.numeric}\NormalTok{(}\FunctionTok{c}\NormalTok{(DF\_TOTAL\_MIN}\SpecialCharTok{$}\NormalTok{prior\_par1\_st1[i], DF\_TOTAL\_MIN}\SpecialCharTok{$}\NormalTok{prior\_par1\_st2[i]))}
\NormalTok{  stepPar0\_p }\OtherTok{\textless{}{-}} \FunctionTok{c}\NormalTok{(par0\_p, par1\_p)}
  \FunctionTok{print}\NormalTok{(stepPar0\_p)}
  \CommentTok{\#op1}
\NormalTok{  m\_cosecha\_p}\OtherTok{\textless{}{-}} \FunctionTok{fitHMM}\NormalTok{(}\AttributeTok{data =}\NormalTok{ dataCosecha, }\AttributeTok{nbStates =} \DecValTok{2}\NormalTok{ , }\AttributeTok{stepPar0 =}\NormalTok{ stepPar0\_p, }\AttributeTok{angleDist =} \StringTok{"none"}\NormalTok{, }\AttributeTok{stepDist =} \FunctionTok{as.character}\NormalTok{(DF\_TOTAL\_MIN}\SpecialCharTok{$}\NormalTok{model[i]))}
\NormalTok{  modelos\_p[[}\FunctionTok{as.character}\NormalTok{(DF\_TOTAL\_MIN}\SpecialCharTok{$}\NormalTok{model[i])]] }\OtherTok{\textless{}{-}}\NormalTok{ m\_cosecha\_p }
\NormalTok{\}}
\end{Highlighting}
\end{Shaded}

\begin{verbatim}
## [1]  2.2  0.8  5.1 13.2
## [1] 1.94 5.61 6.46 1.15
## [1] 0.99 0.05 1.47 1.22
\end{verbatim}

\begin{Shaded}
\begin{Highlighting}[]
\FunctionTok{plot}\NormalTok{(modelos\_p}\SpecialCharTok{$}\NormalTok{gamma, }\AttributeTok{animals =} \FunctionTok{c}\NormalTok{(}\StringTok{"I\_Car12"}\NormalTok{, }\StringTok{"H\_Fran9"}\NormalTok{), }\AttributeTok{ask =} \ConstantTok{FALSE}\NormalTok{, }\AttributeTok{plotTracks =} \ConstantTok{TRUE}\NormalTok{, }\AttributeTok{breaks =} \DecValTok{100}\NormalTok{, }\AttributeTok{col =}\NormalTok{ groupColors2, }\AttributeTok{lwd=}\FloatTok{3.0}\NormalTok{)}
\end{Highlighting}
\end{Shaded}

\begin{verbatim}
## Decoding states sequence... DONE
\end{verbatim}

\includegraphics[width=0.5\linewidth]{analisis_moveHMM_RM_files/figure-latex/unnamed-chunk-22-1}
\includegraphics[width=0.5\linewidth]{analisis_moveHMM_RM_files/figure-latex/unnamed-chunk-22-2}
\includegraphics[width=0.5\linewidth]{analisis_moveHMM_RM_files/figure-latex/unnamed-chunk-22-3}
\includegraphics[width=0.5\linewidth]{analisis_moveHMM_RM_files/figure-latex/unnamed-chunk-22-4}

\begin{Shaded}
\begin{Highlighting}[]
\FunctionTok{plot}\NormalTok{(modelos\_p}\SpecialCharTok{$}\NormalTok{lnorm, }\AttributeTok{animals =} \FunctionTok{c}\NormalTok{(}\StringTok{"I\_Car12"}\NormalTok{, }\StringTok{"H\_Fran9"}\NormalTok{), }\AttributeTok{ask =} \ConstantTok{FALSE}\NormalTok{, }\AttributeTok{plotTracks =} \ConstantTok{TRUE}\NormalTok{, }\AttributeTok{breaks =} \DecValTok{100}\NormalTok{, }\AttributeTok{col =}\NormalTok{ groupColors2, }\AttributeTok{lwd=}\FloatTok{3.0}\NormalTok{)}
\end{Highlighting}
\end{Shaded}

\begin{verbatim}
## Decoding states sequence... DONE
\end{verbatim}

\includegraphics[width=0.5\linewidth]{analisis_moveHMM_RM_files/figure-latex/unnamed-chunk-23-1}
\includegraphics[width=0.5\linewidth]{analisis_moveHMM_RM_files/figure-latex/unnamed-chunk-23-2}
\includegraphics[width=0.5\linewidth]{analisis_moveHMM_RM_files/figure-latex/unnamed-chunk-23-3}
\includegraphics[width=0.5\linewidth]{analisis_moveHMM_RM_files/figure-latex/unnamed-chunk-23-4}

\begin{Shaded}
\begin{Highlighting}[]
\FunctionTok{plot}\NormalTok{(modelos\_p}\SpecialCharTok{$}\NormalTok{weibull, }\AttributeTok{animals =} \FunctionTok{c}\NormalTok{(}\StringTok{"I\_Car12"}\NormalTok{, }\StringTok{"H\_Fran9"}\NormalTok{), }\AttributeTok{ask =} \ConstantTok{FALSE}\NormalTok{, }\AttributeTok{plotTracks =} \ConstantTok{TRUE}\NormalTok{, }\AttributeTok{breaks =} \DecValTok{100}\NormalTok{, }\AttributeTok{col =}\NormalTok{ groupColors2, }\AttributeTok{lwd=}\FloatTok{3.0}\NormalTok{)}
\end{Highlighting}
\end{Shaded}

\begin{verbatim}
## Decoding states sequence... DONE
\end{verbatim}

\includegraphics[width=0.5\linewidth]{analisis_moveHMM_RM_files/figure-latex/unnamed-chunk-24-1}
\includegraphics[width=0.5\linewidth]{analisis_moveHMM_RM_files/figure-latex/unnamed-chunk-24-2}
\includegraphics[width=0.5\linewidth]{analisis_moveHMM_RM_files/figure-latex/unnamed-chunk-24-3}
\includegraphics[width=0.5\linewidth]{analisis_moveHMM_RM_files/figure-latex/unnamed-chunk-24-4}

\begin{Shaded}
\begin{Highlighting}[]
\NormalTok{FIG\_AIC\_model }\OtherTok{\textless{}{-}}\NormalTok{ DF\_TOTAL }\SpecialCharTok{\%\textgreater{}\%} 
  \FunctionTok{filter}\NormalTok{(AIC\_model }\SpecialCharTok{!=} \StringTok{"Inf"}\NormalTok{)}\SpecialCharTok{\%\textgreater{}\%} 
  \FunctionTok{ggplot}\NormalTok{(}\FunctionTok{aes}\NormalTok{(}\AttributeTok{x=}\NormalTok{ model, }\AttributeTok{y=}\NormalTok{ AIC\_model))}\SpecialCharTok{+}
  \FunctionTok{geom\_boxplot}\NormalTok{(}\FunctionTok{aes}\NormalTok{(}\AttributeTok{fill=}\NormalTok{ model))}\SpecialCharTok{+}
  \FunctionTok{geom\_jitter}\NormalTok{(}\FunctionTok{aes}\NormalTok{(}\AttributeTok{x=}\NormalTok{ model, }\AttributeTok{y=}\NormalTok{ AIC\_model), }\AttributeTok{col=} \StringTok{"darkred"}\NormalTok{)}\SpecialCharTok{+}
  \FunctionTok{scale\_fill\_manual}\NormalTok{(}\AttributeTok{values =}\NormalTok{ colorsGris) }\SpecialCharTok{+}
  \FunctionTok{theme\_bw}\NormalTok{()}

\CommentTok{\#ggsave(FIG\_AIC\_model,filename=paste("../output/", "FIG\_AIC\_MODEL\_100", ".png", sep=""),  height = 5, width = 16) \# ID will be the unique identifier. and change the extension from .png to whatever you like (eps, pdf etc)}

\NormalTok{FIG\_AIC\_model\_sinOut }\OtherTok{\textless{}{-}}\NormalTok{ DF\_TOTAL }\SpecialCharTok{\%\textgreater{}\%} 
  \FunctionTok{filter}\NormalTok{(AIC\_model }\SpecialCharTok{!=} \StringTok{"Inf"}\NormalTok{)}\SpecialCharTok{\%\textgreater{}\%} 
  \FunctionTok{filter}\NormalTok{(AIC\_model }\SpecialCharTok{\textgreater{}}\DecValTok{4000}\NormalTok{)}\SpecialCharTok{\%\textgreater{}\%} 
  \FunctionTok{ggplot}\NormalTok{(}\FunctionTok{aes}\NormalTok{(}\AttributeTok{x=}\NormalTok{ model, }\AttributeTok{y=}\NormalTok{ AIC\_model))}\SpecialCharTok{+}
  \FunctionTok{geom\_boxplot}\NormalTok{(}\FunctionTok{aes}\NormalTok{(}\AttributeTok{fill=}\NormalTok{ model), }\AttributeTok{width =} \FloatTok{0.2}\NormalTok{)}\SpecialCharTok{+}
  \FunctionTok{geom\_jitter}\NormalTok{(}\FunctionTok{aes}\NormalTok{(}\AttributeTok{x=}\NormalTok{ model, }\AttributeTok{y=}\NormalTok{ AIC\_model), }\AttributeTok{col=} \StringTok{"darkred"}\NormalTok{, }\AttributeTok{width =} \FloatTok{0.1}\NormalTok{)}\SpecialCharTok{+}
  \FunctionTok{scale\_fill\_manual}\NormalTok{(}\AttributeTok{values =}\NormalTok{ colorsGris)}\SpecialCharTok{+}
  \FunctionTok{theme\_bw}\NormalTok{()}

\CommentTok{\#ggsave(FIG\_AIC\_model\_sinOut,filename=paste("../output/", "FIG\_AIC\_100\_SINOUT\_MODEL", ".png", sep=""),  height = 5, width = 16) \# ID will be the unique identifier. and change the extension from .png to whatever you like (eps, pdf etc)}
\end{Highlighting}
\end{Shaded}


\end{document}
